\section{Technical Description of Bottlenose}

Our web application, Bottlenose, was initially developed to support
the teaching of a ``flipped'' course, where students would watch video
lectures online before class to prepare for classroom questions and
discussion. The design of Bottlenose also called for online submission
and grading of programming assignments, which turns out to be a useful
piece of functionality even for traditional courses.

Bottlenose is built using the Ruby on
Rails\footnote{http://rubyonrails.org/} web application devopment
framework. This framework has allowed the application to be built
rapidly, but also provides built in automated testing infrasture which
should help the application stay high quality and maintainble as it
grows. The application follows standard Rails conventions. A
PostgreSQL\footnote{http://postgres.org/} database to store most
application state, although student submissions are stored on the file
system.

A simple process for online submission of assignments is
provided. Each student is emailed an authentication link which brings
the to their list of assignments and identifies them to the
application. Assignments are submitted by uploading the programming
code directly in their web browser. Both single-file assignments and
more complicated assignments (submitted as a gzipped tarball) are
supported. The automated grading process begins immediately when an
assignment is submitted, giving students feedback within a few
seconds.  Students may attempt submissions multiple times.

In order to automatically grade student programs, submissions are
compiled and run on the server. Allowing students to run arbitrary
code on the server is clearly a potential security issue, so
Bottlenose uses a sandbox mechanism to prevent student programs from
causing problems. Five major techniques are used to isolate student
programs from the rest of the system:

\begin{enumerate}
\item \textbf{Separate system user} - Each student program is run under
  a separate system user with minimal Unix permissions.
\item \textbf{Run in a ``chroot''} - Student programs can only access
  specific, whitelisted parts of the file system.
\item \textbf{Resource limits} - The ``setrlimit'' system call is used to
  set limits on the use of a variety of resources, including RAM, child
  processes, and created file size.
\item \textbf{Isolated working directory} - Each program is executed in
  a separate ``tmpfs'' filesystem which ceases to exist when the grading
  process finishes.
\item \textbf{Watchdog timer} - A grading process is terminated if it lasts
  more than five minutes.
\end{enumerate}

This sandbox mechanism isn't foolproof, but it has worked reasonably
well at preventing the grading server from being disrupted by common
student mistakes like infinite memory allocation loops.
