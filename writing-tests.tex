\section{Writing Tests}

In preparation for student submissions of assignments to Bottlenose, the
instructor writes a set of test scripts which automatically evaluate specific
aspects of students' programs. A test may evaluate, for example, that given
specific input, the program generates proper output; it may check whether
certain required functions are provided by the student's program; it can
determine whether the program exited successfully or crashed; etc.

Tests can be written in any language. The only requirements for tests are that
they conform to the \textit{Test Anything
  Protocol}\footnote{http://search.cpan.org/~petdance/Test-Harness-2.64/lib/Test/Harness/TAP.pod}
(TAP). With TAP, a test outputs an indication of the number of tests that will
be run, e.g., \textbf{1..4} for four tests, followed by a line of output for
each test which indicates whether the test, e.g., test \#2, succeeded
(\textbf{ok 2}) or failed (\textbf{not ok 2}. The tests are otherwise free to
provide input in any form to the program under test and to retrieve output via
their standard output mechanism or via a file. Tests are also free to provide
additional output, such as a description of the test to be run, or student's
and expected output, when a program does not yield the correct results.

Given the flexibility of this testing harness, it is possible to run some
interesting tests on students' programs to let them know that they are
correctly accomplishing their program's goal. While beginning study of heap
allocation using the \texttt{malloc} and \texttt{free} functions in C, for
example, we used a test that compiled the student's program into an object
file, and then used the \textit{objcopy} utility to replace their
\texttt{main} function with one of the test's own, and to replace their calls
to \texttt{malloc} and \texttt{free} with calls to alternate function names
which were defined in the test. In a simple, early heap allocation assignment,
for example, students were to write a program that allocate one integer of
heap storage, assigned the number \textbf{6} into that storage, and then free
the allocated memory. The functions in the test which replaced \texttt{malloc}
and \texttt{free} were then able to ensure that \texttt{malloc} was being
called with the correct size, that the correct address was passed to
\texttt{free}, and that the value \textbf{6} had in fact been put in the
correct memory location by the student's program. Similarly, the replacement
functions were able to test that the student's application correctly handled
an out-of-memory condition as indicated to their program by a NULL return from
\texttt{malloc}.

As we gained more experience writing tests for Bottlenose, we found that we
could easily run student programs under \textit{valgrind} with varying options
to detect different types of student errors. We were able to use \textit{nm}
to look at their object file to confirm that they were providing the specified
functions, and even implement unit testing of those specific functions. For
example, in a bubble sort assignment, the students were required to implement
a \texttt{swapValues} function to swap two values (for practice passing
pointers to integers), and a \texttt{bubblesort} function which was required
to use the \texttt{swapValues} function. The tests for this function were able
to unit test the students' \texttt{swapValues} functions to confirm correct
operation, in addition to testing the bubble sort program for correct output
given various inputs.

%%% where is this discussed?
%Student projects are submitted to Bottlenose by uploading either a single
%source file, or an archive file (gzipped tar format) containing a set of files
%comprising the submission.

